% Typografie a publikování
% 5. projekt
% Juraj Holub
% xholub40@stud.fit.vutbr.cz

\documentclass[usenames,dvipsnames]{beamer}
\usepackage[utf8]{inputenc}
\usepackage[czech]{babel}
\usepackage{tikz}
\usepackage{amsmath, amsthm, amsfonts, amssymb}
\usetheme{Madrid}
\setbeamercolor{alerted text}{fg=BlueViolet}
\usepackage{caption}
\usepackage{subcaption} 
\usepackage{graphicx}
\usepackage{float}
\usepackage{breakurl}

\showboxdepth=\maxdimen
\showboxbreadth=\maxdimen
\newcommand\pro{\item[\textbf{$+$}]}
\newcommand\con{\item[\textbf{$-$}]}

\title{IAL - Algoritmy \\
IFJ - Formální jazyky a překladače \\
projekt}
%\titlegraphic{\includegraphics[height=4cm]{s1mple}}
\subtitle{Tým 008, varianta II}
\date{}

\begin{document}

\begin{frame}
\maketitle
\begin{table}[H]
	\Large
	\centering
	\resizebox{0.6\linewidth}{!}{
	\begin{tabular}{llc}
		\textbf{Člen}        & \textbf{Login} & \textbf{Rozdelenie bodov} \\
		Juraj Holub (vedúci) & \texttt{xholub40}       & 33\%                      \\
		Matej Parobek        & \texttt{xparob00}       & 33\%                      \\
		Samuel Krempaský      & \texttt{xkremp01}      & 33\%                      \\
		Vitalina Koloda      & \texttt{xkolod02}       & 0\%                      
	\end{tabular}
}
\end{table}
\end{frame}

\begin{frame}
	\frametitle{Práca v týme}
	\begin{itemize}
		\item rozdelenie úloh
		\begin{itemize}
			\item lexikálna analýza
			\item precedenčná syntaktická analýza a sémantická anylýza
			\item syntaktická analýza zhora nadol a sémantická anylýza
			\item tabuľka symbolov
			\item generátor kódu
		\end{itemize}
		\item verzovací systém GIT
		\item dohonuté komunikačné kanály
		\item house style kódu
		\begin{itemize}
			\item doxygen komentáre
			\item jeden jazyk (ANJ) pre všetky zdrojové kódy
		\end{itemize}
	\end{itemize}
\end{frame}

\begin{frame}
	\frametitle{Tabuľka symbolov}
	Tabuľka s rozptýlenými položkami (TRP) s explicitným zreťazením synoným.
	\begin{itemize}
		\item veľkosť TRP
		\item reťazec ako kľúč TRP
		\item hash funkcia \textbf{dbj2}\footnote{Zdroj dbj2: 
			\urlstyle{rm}
			\url{http://www.cse.yorku.ca/~oz/hash.html}}
	\end{itemize}
	Obsah položky tabuľky symbolov:
	\begin{itemize}
		\item typ symbolu (premenná, konštanta, funkcia)
		\item datový typ symbolu (neznámy, integer, string...)
		\item identifikátor (kľúč TRP)
		\item pre funkcie počet parametrov
	\end{itemize}
\end{frame}

\begin{frame}
	\frametitle{Lexikálna analýza}
	
	\begin{itemize}
	    
		\item{implementácia konečným  automatom}
		\item \texttt{get\_token()}
		\item \texttt{ret\_token()}
		\item konverzia IFJ18 reťazcov na IFJcode18 reťazce
	\end{itemize}
	Token
	\begin{itemize}
		\item typ lexémy (kľúčové slová, konštanty, premenné...)
		\item atribút lexémy (hodnota konštanty, )
	\end{itemize}

\end{frame}
%	
\begin{frame}
	\frametitle{Syntaktická analýza}
	Metoda rekurzívneho zostupu:
	\begin{itemize}
		\item definície funkcií
	 	\item volanie funkcií
		\item cykly
		\item podmienky
		\item priradenia
	\end{itemize}
	Precedenčná syntaktická analýza:
	\begin{itemize}
		\item logické výrazy
		\item aritmetické výrazy
	\end{itemize}
	Predávanie riadenia:
	\begin{itemize}
		\item \texttt{ret\_token()}
	\end{itemize}
\end{frame}
%
\begin{frame}
	\frametitle{Sémantická anylýza a generovanie IFJcode18}
	Sémantická analýza:
	\begin{itemize}
		\item zvlášť pre každé pravidlo v LL gramatike a precedenčnej tabuľky
		\item tabuľka symbolov zvlášť pre
		\begin{itemize}
			\item hlavné telo programu
			\item každé telo funkcie
			\item deklarácie uživateľských a knihovných funkcií
		\end{itemize}
	\end{itemize}
	Generátor kódu:
	\begin{itemize}
		\item jednosmerne viazaný zoznam inštukcií IFJcode18
		\item premenné definované vždy na začiatku lokálneho rámca
	\end{itemize}
\end{frame}

\end{document}
